\documentclass[journal]{vgtc}                % final (journal style)
%\documentclass[review,journal]{vgtc}          % review (journal style)
%\documentclass[widereview]{vgtc}             % wide-spaced review
%\documentclass[preprint,journal]{vgtc}       % preprint (journal style)
%\documentclass[electronic,journal]{vgtc}     % electronic version, journal
\let\ifpdf\relax

%% Uncomment one of the lines above depending on where your paper is
%% in the conference process. ``review'' and ``widereview'' are for review
%% submission, ``preprint'' is for pre-publication, and the final version
%% doesn't use a specific qualifier. Further, ``electronic'' includes
%% hyperreferences for more convenient online viewing.

%% Please use one of the ``review'' options in combination with the
%% assigned online id (see below) ONLY if your paper uses a double blind
%% review process. Some conferences, like IEEE Vis and InfoVis, have NOT
%% in the past.

%% Please note that the use of figures other than the optional teaser is not permitted on the first page
%% of the journal version.  Figures should begin on the second page and be
%% in CMYK or Grey scale format, otherwise, colour shifting may occur
%% during the printing process.  Papers submitted with figures other than the optional teaser on the
%% first page will be refused.

%% These three lines bring in essential packages: ``mathptmx'' for Type 1
%% typefaces, ``graphicx'' for inclusion of EPS figures. and ``times''
%% for proper handling of the times font family.

% Math-mode symbol & verbatim
\def\W#1#2{$#1{#2}$ &\tt\string#1\string{#2\string}}
\def\X#1{$#1$ &\tt\string#1}
\def\Y#1{$\big#1$ &\tt\string#1}
\def\Z#1{\tt\string#1}

\usepackage{mathptmx}
%%\usepackage{psfrag}
\usepackage{graphicx}
\usepackage{times}
\usepackage{epstopdf}

\usepackage{amssymb,amsmath,amsthm}
\usepackage[lined,linesnumbered]{algorithm2e}
\usepackage{caption}
\usepackage{subcaption}
\usepackage{multirow}
\usepackage{framed}
\usepackage{tikz}
\usepackage{array}

%% We encourage the use of mathptmx for consistent usage of times font
%% throughout the proceedings. However, if you encounter conflicts
%% with other math-related packages, you may want to disable it.

%% This turns references into clickable hyperlinks.
\usepackage[bookmarks,backref=true,linkcolor=black]{hyperref} %,colorlinks
\hypersetup{
  pdfauthor = {},
  pdftitle = {},
  pdfsubject = {},
  pdfkeywords = {},
  colorlinks=true,
  linkcolor= black,
  citecolor= black,
  pageanchor=true,
  urlcolor = black,
  plainpages = false,
  linktocpage
}

%% If you are submitting a paper to a conference for review with a double
%% blind reviewing process, please replace the value ``0'' below with your
%% OnlineID. Otherwise, you may safely leave it at ``0''.
\onlineid{0}

%% declare the category of your paper, only shown in review mode
\vgtccategory{Technique}

%% allow for this line if you want the electronic option to work properly
\vgtcinsertpkg

%% In preprint mode you may define your own headline.
%\preprinttext{To appear in an IEEE VGTC sponsored conference.}


%%%%%%%%%%%%%%%%%%%%%%%%%%%%%%%%%%%%%%%%%%%%%%%%%%%%%%%%%%%%%%%%%%%%%%%%%%%%%
%
% Math commands
%
%%%%%%%%%%%%%%%%%%%%%%%%%%%%%%%%%%%%%%%%%%%%%%%%%%%%%%%%%%%%%%%%%%%%%%%%%%%%%

\newcommand {\emath}[1]  {\ensuremath{#1}}
\newcommand {\R}         {\emath{\mathbb{R}}}        % Real space
\newcommand {\Real}[1]   {\emath{\mathbb{R}^{#1}}}   % Real space
\newcommand {\Rd}        {\Real{d}}                  % R^d
\newcommand {\Rdone}     {\Real{d+1}}                % R^d+1
\newcommand {\Rk}        {\Real{k}}                  % R^k
\newcommand {\Rtwo}      {\Real{2}}                  % R^2
\newcommand {\Rthree}    {\Real{3}}                  % R^3
\newcommand {\Rfour}     {\Real{4}}                  % R^4
\newcommand {\Sphere}[1] {\emath{\mathbb{S}^{#1}}}   % Sphere
\newcommand {\Sk}        {\Sphere{k}}                % S^k
\newcommand {\Sd}        {\Sphere{d}}                % S^d
\newcommand {\BB}        {\emath{\mathbb{B}}}        % B
\newcommand {\Ball}[1]   {\emath{\mathbb{B}^{#1}}}   % B^{#1}
\newcommand {\Ballep}    {\emath{B^{\epsilon}_{p}}}  % B^e_p
\newcommand {\cl}        {\emath{\mathrm{cl}}}       % cl
\newcommand {\cb}        {\emath{\mathbf{c}}}        % bold c
\newcommand {\eb}        {\emath{\mathbf{c}}}        % bold e
\newcommand {\tpi}       {\emath{\tilde{\pi}}}       % \pi~
\newcommand {\gDim}[1]   {\emath{#1 \times #1 \times #1}} % DxDxD

\newcommand {\tg}        {\emath{\tilde{g}}}
\newcommand {\tn}        {\emath{\tilde{n}}}
\newcommand {\IV}        {\emath{\mathcal{I_V}}}
\newcommand {\Orth}      {\emath{\mathcal{O}}}
\newcommand {\hN}        {\emath{\widehat{N}}}
\newcommand {\XX}        {\emath{\mathcal{X}}}

\newtheorem{proposition}{Proposition}
\newtheorem{corollary}{Corollary}[proposition]
\newtheorem{lemma}[proposition]{Lemma}


%%%%%%%%%%%%%%%%%%%%%%%%%%%%%%%%%%%%%%%%%%%%%%%%%%%%%%%%%%%%%%%%%%%%%%%%%%%%%
%
% tikz block styles
%
%%%%%%%%%%%%%%%%%%%%%%%%%%%%%%%%%%%%%%%%%%%%%%%%%%%%%%%%%%%%%%%%%%%%%%%%%%%%%

\usetikzlibrary{shapes,arrows}

\tikzstyle{action} = [rectangle, draw, text centered, node distance=4cm, minimum height=4em]
\tikzstyle{source} = [draw, ellipse, text centered, node distance=1.5cm, minimum height=4em, text width=3cm]
\tikzstyle{sink} = [draw, ellipse, text centered, node distance=2cm, minimum height=4em]
\tikzstyle{line} = [draw, -latex']
\tikzstyle{figlabel} = [text centered, node distance=1.5cm, text width=1cm]

%%%%%%%%%%%%%%%%%%%%%%%%%%%%%%%%%%%%%%%%%%%%%%%%%%%%%%%%%%%%%%%%%%%%%%%%%%%%%
%
% algorithm2e keywords and commands
%
%%%%%%%%%%%%%%%%%%%%%%%%%%%%%%%%%%%%%%%%%%%%%%%%%%%%%%%%%%%%%%%%%%%%%%%%%%%%%

% algorithm2e global keywords
\SetKw{Function}{Function}
\SetKw{true}{true}
\SetKw{false}{false}
\SetKw{KwAnd}{and}
\SetKw{KwOr}{or}
\SetKw{true}{true}
\SetKw{false}{false}
\SetKw{KwElse}{else}
\SetKw{KwDownTo}{downto}
\SetKwData{NULL}{NULL}
\SetKwInOut{Input}{Input}
\SetKwInOut{Output}{Output}
\SetKwInOut{Result}{Result}
\SetKwInOut{Requires}{Requires}
\ResetInOut{Requires1}
\SetKwComment{NoLineNum}{}{}
\SetCommentSty{textit}
\SetArgSty{textrm}
\SetFuncSty{textsc}
\SetAlgoLined

\IncMargin{1ex}

\SetKwFunction{AngleTest}{AngleTest}
\SetKwFunction{ScalarTest}{ScalarTest}
\SetKwFunction{ReliableGrad}{ReliableGrad}
\SetKwFunction{MergeSharp}{MergeSharp}
\SetKwFunction{FindSharp}{FindSharp}
\SetKwFunction{CountDegree}{CountDegree}
\SetKwFunction{SelectiveFindSharp}{SelectiveFindSharp}
\SetKwFunction{Magnitude}{Magnitude}
\SetKwFunction{Angle}{Angle}
\SetKwFunction{Distance}{Distance}
\SetKwData{Grid}{Grid}
\SetKwData{numAgree}{numAgree}
\SetKwData{errorDist}{errorDist}
\SetKwData{maxErrorDist}{maxErrorDist}
\SetKwData{numIter}{numIter}

% Algorithm function names and variables
\SetKwFunction{DoesOrthMatch}{DoesOrthMatch}
\SetKwFunction{DoesOrthMatchA}{DoesOrthMatchA}
\SetKwFunction{DoesOrthMatchB}{DoesOrthMatchB}
\SetKwFunction{ExtendReliable}{ExtendReliable}


\SetKwData{Center}{Center}
\SetKwData{Centroid}{Centroid}
\SetKwData{isovLoc}{isovLoc}
\SetKwData{numLargeEigenvalues}{numLargeEigenvalues}

% algorithm2e reset line number
\newcommand {\ResetAlgoLineNumber} {\setcounter{AlgoLine}{0}}

\SetAlgoCaptionSeparator{.}




\title{Visualizing flow fields using fractal dimensions}

%% This is how authors are specified in the journal style

%% indicate IEEE Member or Student Member in form indicated below
\author{Ross Vasko, Han-Wei Shen and Rephael Wenger}
\authorfooter{
The Ohio State University. E-mail: vasko.38@osu.edu, shen.94@osu.edu
 and wenger.4@osu.edu}

\abstract{
Streamlines are a popular way of visualizing flow in vector fields.
A major challenge in streamline visualization is selecting the streamlines
for visualization.
Rendering too many streamlines clutters the visualization and makes
features of the field difficult to identify. 
Rending too few streamlines causes viewers to completely miss features
of the flow field not rendered. 

The fractal dimension of a streamline represents its space-filling properties.
To identify complex or interesting streamlines, 
we build a regular grid of scalar values 
which represent the fractal dimension of streamlines around each grid vertex.
High fractal dimension indicates vortices or turbulent regions.
We use this scalar grid both to filter streamlines by fractal dimension
and to identify and visualize regions containing vortices and turbulence.
We describe an interactive tool which allows for quick streamline selection
and visualization of regions containing vortices and turbulence.
}

%% Keywords that describe your work. Will show as 'Index Terms' in journal
%% please capitalize first letter and insert punctuation after last keyword
\keywords{Streamlines, fractal dimension.}

%% ACM Computing Classification System (CCS). 
%% See <http://www.acm.org/class/1998/> for details.
%% The ``\CCScat'' command takes four arguments.

\CCScatlist{ % not used in journal version
\CCScat{I.3.5}{Computer Graphics}{Computational Geometry and Object Modeling}
}

%% Uncomment below to include a teaser figure.
%\teaser{
%}

%% Uncomment below to disable the manuscript note
%\renewcommand{\manuscriptnotetxt}{}

%% Copyright space is enabled by default as required by guidelines.
%% It is disabled by the 'review' option or via the following command:
% \nocopyrightspace

\renewcommand{\textfraction}{0.2}
\renewcommand{\dbltopfraction}{0.8}	
\renewcommand{\topfraction}{0.8}	


%%%%%%%%%%%%%%%%%%%%%%%%%%%%%%%%%%%%%%%%%%%%%%%%%%%%%%%%%%%%%%%%
%%%%%%%%%%%%%%%%%%%%%% START OF THE PAPER %%%%%%%%%%%%%%%%%%%%%%
%%%%%%%%%%%%%%%%%%%%%%%%%%%%%%%%%%%%%%%%%%%%%%%%%%%%%%%%%%%%%%%%%

\begin{document}

\firstsection{Introduction}

\maketitle

\section{Related Work}

\section{Box Counting Ratio}


\section{Streamline Complexity Grid}

For each regular point $p$ in a vector field $f: \Rthree \Rightarrow \Rthree$,
let $\zeta_p$ be the unique streamline passing through $p$.
Define $\phi_w(p)$ as the local box counting ratio of $\zeta_p$
in a $\gDim{w}$ region around $p$,
where boxes have edge lengths 1 and 2.
Function $\phi_w$ defines a scalar field on the regular points of $f$.
The scalar $\phi_w(p)$ represents the complexity of the streamline
through $p$ in a $\gDim{w}$ neighborhood of $p$.
We call the scalar field $\phi_w$, the streamline complexity field of $f$.

To construct a scalar grid representing the streamline complexity
field of $f$,
we compute a set of streamlines
so that every voxel is intersected by at least one streamline.
We then compute the box counting ratio around each voxel $v$,
by selecting a streamline $\zeta_v$ intersecting the voxel $v$
and computing the local box counting ratio of $\zeta_v$
in a $\gDim{w}$ region centered at $v$.
We call the resulting scalar grid,
the streamline complexity grid of $f$.

To compute a set of streamlines intersecting every voxel, we... %find exactly how the streamlines are being computed and stored

To compute the local box counting ratio of $\zeta_v$ 
in a $\gDim{w}$ region centered at $v$, we... %find exactly how the box counting dimensions are done with the cutoffs and percentiles


\section{Visualization Techniques}

The streamline complexity computations can be used in a variety of ways to filter streamlines of varying complexities or to provide a method of visualization for the flow field.

\textbf{Streamline filtering by value:}
The user is able to choose two constants, $a$ and $b$ where $a < b$, and only display streamlines with a complexity inbetween the chosen values. 
Specifically, the streamline $\zeta_p$ will be displayed only if $a \leq \phi_w(p)$ and $\phi_w(p) \leq b$.
The constants allow the user to choose the level of streamline complexity that they will view.
By choosing constants of values near 1, streamlines with a low box counting ratio and smooth flow will be displayed.
By choosing constants of values above 2, streamlines with a relatively high box counting ratio and turbulent or complex flow will be displayed.

\textbf{Local maximums:}
A significant amount of clutter in the streamline display will remain if additional filtering methods are not considered.
Several streamlines in the visualization will be visuallly similar or provide redundant information.
We are able to only show the local maximum streamlines to filter streamlines that all represent a single region or feature of the flow.
Local maximum filtering will show a streamline $\zeta_p$ only if for all of the 8 points $q$ that directly neighbor $p$ on the complexity grid, $\phi_w(p) > \phi_w(q)$.
This method allows for single streamlines representatives to be shown for each feature or region rather than several, cluttered streamlines.

\textbf{Colored plane:}
A color plane can be used to allow the user to visualize the scalar complexity grid, $\phi_p$, directly.
A color gradient from blue to green to red is able to be defined and mapped to values in the range 0 to 3, for each of the possible box counting dimensions ratios.
Low scalar values will be displayed as blue colors, while high scalar values will be displayed as red colors.
A plane is then defined on the scalar complexity grid and each point on the plane is colored from this defined color gradient.
The user is able to control the plane through the scalar complexity grid to identfiy regions varying complexity in the grid.
Once regions of interest are identified through the color plane, the user can display streamlines near that region to understand its behavior.

\textbf{Isosurfaces:} 
An isosurface can be used to highlight regions of the flow field with a high complexity.
An isosurface, which is defined by
\begin{equation} \{ x \mid \phi_w(x) = \sigma \}\end{equation}
where $\sigma$ is the isovalue and $\phi_w$ is the streamline complexity field, will seperate all scalar values above $\sigma$ from the values below $\sigma$ on the streamlines complexity.
This isosurface will enclose the regions of streamlines with a fractal dimension higher than the $\sigma$ value and provide a simple way to identify regions of a defined complexity.
At particularly high $\sigma$ values, the isosurface will only enclose vortices and turbulent features of the flow field that the user may have otherwise missed.

\textbf{Gradient magnitudes:}
The gradient magnitudes of the streamline complexity grid can be calculated to create a new gradient magnitude scalar grid $\phi_g$.
The scalar $\phi_g(x)$ is given by $\| \nabla \phi_p(x) \|$.
Another isosurface can be used to visualization the function $\phi_g$ to identify regions of high change of turbulence or complexity.
Vortices in the flow field tend to have high complexity values recorded near their centers, with values quickly decreasing towards their boundaries.
When a high isovalue is chosen for the gradient magnitude isosurface's isovalue, the isosurface will highlight these isolated regions of turbulence or turbulent regions that quickly become smooth.

\section{Examples}

\section{Evaluation}

\section{Summary and Future Work}

\end{document}
